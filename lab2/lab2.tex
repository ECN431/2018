\documentclass[12pt,a4paper]{article}


\usepackage{hyperref}
\usepackage{graphicx}
\usepackage{multirow}
\usepackage{amsmath}
\usepackage{amssymb}
\usepackage[top=1in, bottom=1in, left=1in, right=1in]{geometry}


\title{ECN431\\Lab session 2\\Empirical market analysis}
\date{Aud 12 \\ Friday, February 2\textsuperscript{nd} 2018}
\begin{document}
\maketitle

\section*{Before the Lab}
\begin{itemize}
\item Obtain the datasets and the example R notebook from the GitHub repository of the course.
\item Try to work through the exercises on your own prior to the lab.
\end{itemize}

\section*{Purpose of this lab}
Learn to apply instrumental variables (IV) methods to real-world data, interpret the results -- in particular for the purpose of estimating competitive supply and demand -- and learn about the electricity market.

\subsection*{Learning goals}
\begin{itemize}
\item	Improve the skill of presenting data for own and other's understanding
\item	Understand what the standard IV method does
\item	Be able to perform IV estimates using R
\item	Know several ways of assessing instrumental variables
\item	Know the basic features of the Nordic electricity market
\item	Be able to estimate supply and demand in competitive markets
\item	Be able to explain, present and identify critical assumptions in your analysis
\end{itemize}


\subsection*{Practical information}
The example R notebook (which you either pulled to your fork from the main repository or downloaded from GitHub) is opened in RStudio. This file contains several commands that you will need in order to answer the questions in these exercises, in addition to information on new commands. Remember: If/when you are uncertain about the purpose, functionality and extensions to/options for some of the commands, remember that you can get documentation for the command in RStudio by pressing F1. You can also try to google your particular problem.



\section*{Exercises}

\subsection*{Warming up at the Fulton fish market}
Use the data set in \emph{fishmarket.dta} for this exercise. The Fulton fish market resides in New York, and was located by Fulton Street in the Financial District, Lower Manhattan for 193 years (until 2005, when it moved to Hunts Point in the Bronx). The sellers at the market buy their fish from suppliers, typically having the fish transported to the market by truck. Buyers on this market typically request large quantities of fish (restaurants and other businesses requiring fish).

The data contain daily quantities and prices for sales of Whiting for one particular seller from December 8th 1991 to May 8th 1992.\footnote{The data was collected by Kathryn Graddy for her PhD dissertation at Princeton University. Her analysis and further description of the data can be read in the article ``Testing for imperfect competition at the Fulton fish market'', published in the RAND Journal of Economics, Vol. 26, No. 1 (pp. 75--92), 1995.} The market is open from Monday to Friday, and prices are to a large degree determined by the supply of Whiting on a given day. Difficult conditions at sea, such as waves and strong wind, will reduce catch. You will apply this insight for the purpose of estimating demand.

\begin{enumerate}
 \item[] \emph{Present and understand your data}
 \item Make a table of summary statistics for price, quantity, and indicators for stormy and mixed weather.
 \item Create the following two graphs: i) Quantity sold per date, and ii) Price per unit per date. Choose a suitable graph type.
 \item Regress quantity on price, and comment. Would you interpret the price coefficient as the price sensitivity of demand? Why or why not?
 \item[] \emph{Assessing and understanding IV}
 \item Create a graph showing price over time, clearly indicating periods where weather at sea is stormy or mixed (preferrably so one can separate the periods where it is stormy or mixed). Comment on the relationship between price and weather conditions.
 \item Regress price on stormy weather and quantity on stormy weather. What is the interpretation of the coefficients? Make a table of the average quantity and price according to whether the sea is stormy or not. Calculate the average change in price and quantity from stormy weather. Also calculate the ratio $\Delta q / \Delta p$ (ratio of the changes).
 \item Regress quantity on price, using the indicator for stormy weather to instrument price. Explain the relation to the ratio of changes you calculated previously.\footnote{If you're stuck, review the lecture slides for instrumental variables, slide 25 and 27.} Discuss the validity of this instrument with your group -- are there any potential issues you could test with the data you have?\footnote{The relevance criterion means that the instrument must affect the endogenous variable. In practice, the this is usually assessed by the instrument having a significant coefficient in the first stage (regression of the endogenous variable on the instrument and controls), as well as the F-statistic for the first stage regression having a value of at least 10 (the F-statistic is the basis of a test for joint statistical significance of all regressors). This is purely a rule of thumb, and higher values are considered better, though in certain cases a lower F-statistic could be used if one has few observations, includes many insignificant controls, and the instrument can otherwise be shown to have a strong correlation with the endogenous variable (though be wary if you see/get low F-statistics, especially if combined with marginal significance of the coefficient on the instrument).}
 \item Regress quantity on indicators for day of week. Do the same for stormy weather. Now perform the previous IV estimates including these indicators (make sure also report the first-stage regression). Discuss the changes in the regression results with your group.
 \item Extra:
 \item[] \emph{Estimating supply and demand}
  \begin{enumerate}
  \item So far, we have been interested in the demand side. Day of week seems to shift demand, particularly Tuesday and Wednesday. Create an indicator that takes the value 1 if the day is either Tuesday or Wednesdays. Regress i) price and ii) quantity on stormy weather and the new indicator for low-demand days. Discuss estimating supply either as the quantity supplied as a function of price, or as the inverse supply schedule (price as a function of quantity), in both cases using low-demand period as an instrument for demand.
  \item Do the IV regression of price on stormy weather and quantity, using low-demand days as an instrument for quantity. Discuss the findings, particularly the validity of the instrument (assess relevance from the first stage) and why or why not you think this supply function seems reasonable. If we believe this supply curve is the right one, what does it tell us about supply in this market?
  \item In a price-quantity diagram, draw the supply function both when the weather is stormy and not, and demand during regular and low-demand days (use the estimated coefficients). Try to use R commands for drawing functions to make the graphs.
  \end{enumerate}
\end{enumerate}



\subsection*{Nord pool spot: The electricity wholesale market}
Electricity markets are special for several reasons. One is the typically massive investments required to have an operative grid for transmission of electricity from where it is produced, to where it is consumed. Another is that supply and demand must be equated in each instant, otherwise the frequency (in the oscillations of the alternating current (AC)) in the grid will fluctuate, potentially damaging electrical apparatus or even leading to transmission halting completely (blackouts). The operation of modern electric power distribution is a feat of engineering (and to some extent economics). In the Nordic countries, the grid and production is separated, where the grid is subject to natural monopoly regulation, while production is ``freely'' competing.\footnote{Production is also subject to a wide range of regulations, which you can read more about from other sources.}

Nord Pool is an electricity market, jointly owned by the Nordic and Baltic transmission system operators. The integrated market covers Denmark, Norway, Sweden, Finland, Belarus, Latvia and Lithuania. Most notably, they operate the day-ahead and the intraday market, both designed as auctions. The day-ahead market sets the plan for production and consumption for the upcoming day, setting prices in the process. The intraday market is an adjustment market where actors can buy and sell production after the day-ahead market has closed, but before the hour of production. There is also a market for financial products related to the Nordic and Baltic power market (power derivatives), operated by NASDAQ OMX Commodities Europe. In addition, large firms might have bilateral contracts for electricity supply with producers. In this exercise, we will study the day-ahead market, which is crucial for the formation of the price of electricity, also in the other markets.

In the day-ahead market, sellers and buyers of power in the whole market area submit their bid schedules, The bids can be set very flexibly, and a bidder can submit bids for both selling and buying (typically turning from buyer to seller at a certain price). The typical actors are power production units on the seller side, and industrial firms with power-intensive production and electricity retailers (selling to consumers and other firms) on the buyer side. The bidders submit separate bids for every hour of the day in which they wish to take part in the market. The day before, bids for each of the 24 hours of the next day are submitted with a deadline at noon.

The auction is a uniform price action, where there will be one price for every unit exchanged in a given hour. The price is chosen such that quantity sold is equal to quantity purchased, where buying bids with the highest submitted price and selling bids with the lowest submitted price are activated first. This is done for the market as a whole in a first step, resulting in what is referred to as the ``system price'' and ``turnover at system price''. Due to transmission constraints between different areas, this result might not be feasible, which will result in prices in some price areas deviating from the system price to make local demand and supply satisfy the transmission constraints.\footnote{There are many ways to set the area prices such that transmission constraints are satisfied, though the algorithm employed searches for the solution which can be thought of as minimizing the shadow prices on the transmission constraints in the problem of maximizing market surplus (maximizing the sum of difference between purchasing and selling bids).}

The single largest power source in the market is hydro, accounting for about 50\% of demanded energy in the Nordic countries in years with normal rain and snow, while nuclear and different types of thermal power (coal, oil, gas, and biofuels) accounts for around 25\% each, while wind power has around 3\% share after an increase in Danish production the last two years.

Due to the local climatic conditions of the Nordic area, there is a clear relationship between energy consumption and the season, driven by temperature. Temperatures within the variation we see in the region will not have an effect on power production.\footnote{It's conceivable that thermal power plants might be subject to regulation during periods of extreme heat, as has been the case in continental Europe, due to waste water of high temperature potentially damaging river ecosystems. As far as I'm aware, this has not been a problem in the Nordics.} In the data set \emph{elmarket.csv}, you will have hourly system prices and turnover at system prices, in addition to rich data on temperatures in Norway and Sweden which you will use to estimate the supply curve.

Hydro power production is heavily dependent on access to rain and melt water to fill up the reservoirs, and it is expected that the amount of water in the reservoirs will affect prices. However, the problem of a producer is not as simple as looking at the amount of water in the reservoir to make a decision on how to set up a bid schedule. If we think about production cost in hydro, the physical costs of producing one unit is basically zero. The true marginal cost comes from something called the \emph{shadow value of water}, an alternative cost which can be thought of in terms of the price one could get by saving the water today, and rather producing tomorrow.\footnote{This is an example of a dynamic optimization problem.} This alternative cost arises because the access to water for production is not without limit, thus not driving prices to zero. In the data, you will also have the power potential of the stored water at weekly levels, and attempt to use this as a supply shifter when estimating demand.

\subsubsection*{Data}
\begin{itemize}
 \item time -- date and hour of day for the observation
 \item price -- the system price from the day-ahead market in EUR/MWh (Euro per 1000 KWh) for each hour
 \item volume -- the turnover at system price from the day-ahead market in MWh for each hour
 \item hydrores -- the total production possible in TWh with the water stored in hydro power reservoirs in the Nordic countries. Measured weekly, on Monday of each week (repeated values from Monday 0 hours to Sunday 23 hours).
 \item Arendal--Uppsala - air temperature in Celsius for several Norwegian and Swedish cities, measured hourly.
\end{itemize}


\subsubsection*{Tasks}
\begin{enumerate}
 \item[] \emph{Data wrangling}
 \item Generate a population-weighted average temperature for each date and hour for the Norwegian areas and the Swedish areas separately, i.e., a Norwegian average temperature and a Swedish average temperature. Use the population figures quoted in the two files \emph{norpop.csv} and \emph{swepop.csv}.
 \item[] \emph{Present and understand your data}
\item Create a table of descriptive statistics for price, volume and magazine capacity, another for temperatures in the Norwegian areas, and a third for temperatures in the Swedish areas.
\item Create the following graphs: i) Average hourly volume per week, ii) Volume weighted average price per week, iii) Average temperature in Norway and Sweden per week, and iv) Hydro reservoir capacity per week (this variable is recorded weekly, so just take the average). Choose the type of graph you think is best for conveying the information. Maybe some of the graphs would make sense to display in the same figure?
\item Create the following graphs: i) Average volume per hour of day, ii) Average volume weighted price per hour of day, iii) Average temperature in Norway and Sweden per hour of day, and iv) volume weighted price per hour of day against average volume per hour of day.\footnote{The last one has 24 points in total -- the volume weighted price for the first hour (average across all days in the sample) against the average volume for the first hour, and so forth.} Choose the type of graph you think is best for conveying the information. Do you choose the same as for the previous plots?
\item Discuss the relationship between electricity use, prices, temperatures, and hydro reservoir reserves. Note in particular the pattern over time, and throughout the day. Discuss in terms of demand-side and supply-side factors.
\item To understand better what is driving electricity use (particularly the pattern we see in volumes over time), make a plot featuring the original series of volumes, as well as a residualized version after regressions of volume on i) temperature in Norway, ii) temperature in Norway and indicators for month, and iii) temperature in Norway, indicators for month and indicators for year. Add back the average volume to the three residualized versions to make the graphs more comparable.\footnote{Try to make a plot featuring the original series in it's own graph above, while the three residualized versions are shown together in one graph below.} Discuss what this figure tells you.
\item To understand what drives hydro reserves, do the same as above; make a plot featuring the original series, as well as residualized versions, after controlling for the same variables as above. How do you think hydro reserves are determined, particularly comparing the raw series of prices and the pattern of hydro reserves with different controls?
\item[] \emph{Estimating supply and demand}
\item You want to estimate the supply curve. Perform an IV-regression of volume on prices, using temperature as instrument, choosing which controls to add. Could hour of day be used as an instrument? Discuss the instrument, the first stage and the results. Also calculate the elasticity of supply with respect to price (at the average volume and price in the sample).
\item To estimate the demand curve, perform an IV-regression of volume on prices, using hydro reserves as instrument. Choose controls to add to the regression. Discuss the instrument, the first stage and the results. Also calculate the elasticity of demand with respect to price (at the average volume and price in the sample). Output both the estimated demand and supply functions side by side in one table.
\item[] \emph{Assessing instruments}
\item To assess the relationship between volume and temperature, graph volume for different \emph{bins} of temperature in Norway using \texttt{ggplot} and \texttt{stat\_summary\_bin} (use the argument \texttt{bins = x} to specify the creation of \emph{x} bins, for instance 100).\footnote{This is a type of discretization, where we choose a range of x-axis values, say, temperatures between -3 and -2, -2 and -1, etc. and calculate the average y-axis value within this range, say, the average temperature between -3 and -2 (and so forth). The command \texttt{stat\_summary\_bin} automates this.} Does the relationship look linear? Also use indicators for each hour as a control (regression on dummies for each value of hour). Discuss the difference. Do the same for price. Feel free to check if the variables in logarithm looks better.
\item Perform the following, relatively complex operations, related to what is called a Visual Instrumental Variables (VIV) plot:
\begin{enumerate}
 \item Generate categories/bins of temperature in Norway in steps of one from the whole degree below the minimum to the whole degree above the maximum.\footnote{You can use \texttt{stat\_summary\_bin} with the argument \texttt{binwidth}, to achieve this.} Make graphs showing the average volume and price within each bin, as well as the number of observations within each bin. Also plot the average volume and price within each bin against each other.\footnote{This last graph is known as the VIV plot, and can be used to assess the instrument in several ways -- both how ``noisy'' the relationship between the two variables are when ``filtered'' through the instrument, and at least speculate about possible functional form issues (if we trust the instrument). The plots for the dependent variable and the suspected endogenous regressor against bins of the instrument correspond to the reduced form and the first stage, and can be used for similar assessments.}
 \item Do the same as above, but first residualize temperature, volume and price with respect to any controls you believe should be included in the supply equation. Add back the average value of each variable to the residuals to make them comparable to the original variables.
 \item Make the same graphs as before, only this time using bins of hydro reserves. Since hydro reserves are only reported weekly, there are only slightly more than 200 effective observations of hydro reserves in the sample, so try with about 13 bins.
 \item Do the same as above, but first residualize hydro reserves, volume and price with respect to any controls you believe should be included in the \emph{demand} equation. Add back averages.
\end{enumerate}

\item Extra:
\begin{enumerate}
\item \emph{Evalutate the approximation:} Discuss and try to check for which range of outcomes you believe the results are good. Can you assess the amount of variation the instrument effectively creates in the outcomes we're interested in here, for example from some of the graphs you've created?
\item[] \emph{Apply the results}
\item The government currently collects a tax of roughly 18 EUR/MWh on electric power.\footnote{This applies in Norway, and there are reductions for several types of actors, but you can disregard these additional layers of complexity for this task.} What is the average effect on prices and volumes if this is increased to 20 EUR/MWh? What is the reduction in consumer surplus and producer surplus, and what is the deadweight loss?
\item Statnett, together with a partner on the German side, has started construction of a new subsea cable to Germany named \emph{NordLink}. The maximum transfer capacity of the cable is 1400 MW. Assume that the cable will increase electricity exports to Germany by 1200 MWh in every hour.\footnote{This is an unrealistically high number, and only serves as an example. The net transfer over the first year of operation is calculated to be 0, and realistic gain/loss calculations would be more complicated than what I'm asking for here.} Calculate the effect on prices and domestic production (and/or domestic consumption). How much does domestic producers gain, and how much does domestic consumers lose from this cable?
\item \emph{Further research:} Go to the webpages of Nordpool spot and download one of sheets containing data on the market cross for a representative day of the sample. Use the bid curves in a suitable hour (e.g., at noon) to calculate the (arc) elasticity of demand and supply (for some area around the equilibrium price). How does your finding here match up with the implied elasticity from your estimates?
\end{enumerate}
\end{enumerate}

%
% \subsection{The Nordic electricity market: Hydro, intertemporal decisions and other issues}
%
% The file \emph{spotprice.dta} contains data on electricity spot price contracts in Norway for 2010--2015. Use these data to answer the following questions.
% \begin{enumerate}
% \item Find the largest and smallest values of the fixed fee for the contract. Does the range make sense? Explain.
% \item Generate a new variable containing the natural logarithm of the fixed fee. Why do we get so many missing values?
% \item How many observations have no fixed fee? What percentage is this of the total sample?
% \item How many observations have a fixed fee between 200 and 400?
% \item Compare the fixed fees and the markups. Which is more often zero?
% \item Find the correlation between fixed fees and markup. What do you conclude?
% \item Make a scatter plot of fixed fees against markups. Does it fit with what you suspected from the correlation? Do you feel that the scatter plot conveys the information in this data set well?
% \item Find the average and standard deviation of markup for contracts without a fixed fee. Would you say that there is wide variation in markups for contracts without fixed fee? Make a histogram of the markup for these contracts. What do you think now?
% \item Find the median fixed fee and markup. If the contract had these terms, and a consumer uses 800 kWh during a month, what would the company get in revenue from this consumer that month? How many contracts have these terms?
% \item Advanced:
% \begin{enumerate}
%  \item Transform the dataset so it contains the number of active firms, number of contracts, the average markup and average fee by week. (One row per week, four columns)
%  \item Create and store two graphs: one for the average markup over time, and one for average fixed fee over time. Insert the graphs into a document.
%  \item What is the correlation between average markup and average fixed fee? Why has this changed so dramatically?
%  \item Make a scatterplot of fixed fees against markups. What do you make of this?
%  \item In a single graph, show both the average markup and number of active firms over time. Do the same for average fixed fee and number of offered products. Do you see any relationship? Do you think it is spurious?\footnote{If you don't know what a spurious relation is, you can visit \url{http://tylervigen.com/spurious-correlations}.}
% \end{enumerate}
% \end{enumerate}
%
%
% The dataset \emph{airfare.dta} contains information on U.S. domestic city-pair flight routes. Load this file into Stata and do the following exercises:
%
% \begin{enumerate}
% \item How many unique routes are there in the data?
% \item Generate the log (natural) of average passengers per day, average one-way fare and distance. Remember to label your variables!
% \item Regress passengers on fare, then regress the log of passengers on log of fare. Interpret the regression coefficients in each model. Which model do you think is better, and why?
% \item Regress log fare on market share of the biggest carrier. Does the result make sense?
% \item Make a scatter plot of log fare on biggest market share. Though this is not too bad, with many observations, scatterplots generally become muddled and useless. Install the command \texttt{binscatter}, and make a binned scatter plot of the log fare against market share of biggest carrier.\footnote{I advice you to read more about this very useful graphing tool at the webpage of Michael Stepner, who created the code: \url{https://michaelstepner.com/binscatter/}}
% \item Make a binned scatter plot of biggest market share against log distance, and another of log fare against log distance. How can this be used to explain the the result from the regression of (log) price on market share?
% \item What regression do you want to run now? Do it, and interpret the results.
% \item Try to find some evidence of (simple) non-linearities in the effect of log distance on (log) fare. Does this affect the coefficient on market share in the fare regression?
% \item Regress log passengers on log fare and some specification of distance you think makes sense. Does the coefficient on log fare make sense? What could be the problem here?
% \item Export the previous table to a document. If you have time, install the package \texttt{estout} and see if you can use the command \texttt{esttab} to format the table directly from Stata.
% \item Advanced:
% \begin{enumerate}
%  \item Regress log passengers on log fare and distance terms, with fixed effects for route (panel method: fixed effect regression). What does the change in the coefficient on log fare from the previous (OLS) regression tell you? Why do the distance terms drop out of this regression?
%  \item Regress log passengers on log fare and some distance specification, while using biggest market share as an instrument for price (e.g., 2SLS). Don't use fixed effects here. Compare the coefficient on log fare to the previous models. Look at the first stage regression, and assess the relevance of the instrument. Do you think the market share is a good instrument? Can you think of any potential problems?
% \end{enumerate}
% \end{enumerate}


\end{document}
